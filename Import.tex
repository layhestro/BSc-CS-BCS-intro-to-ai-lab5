%===============================================================================
% Basic Document Setup
%===============================================================================
\usepackage[T1]{fontenc} 
\usepackage[utf8]{inputenc}
\usepackage{parskip}
\usepackage{geometry}
\usepackage{setspace}
\setstretch{0.93}
% Author information
\author{Grégoire Mariette}
\date{April 2025}

% Page geometry
\geometry{
    a4paper,
    total={180mm, 265mm},  % Increased from 170mm, 257mm
    left=17.5mm,           % Reduced from 20mm
    top=17.5mm,            % Reduced from 20mm
}

%===============================================================================
% Math Packages
%===============================================================================
\usepackage{amsmath}
\usepackage{amsthm}
\usepackage{amssymb}
\usepackage{mathtools}
\usepackage{algorithm}
\usepackage{algorithmic}

% Define theorem-like environments
\theoremstyle{plain}
\newtheorem{theorem}{Theorem}[section]
\newtheorem{lemma}[theorem]{Lemma}
\newtheorem{corollary}[theorem]{Corollary}
\newtheorem{proposition}[theorem]{Proposition}

\theoremstyle{definition}
\newtheorem{definition}{Definition}[section]
\newtheorem{problem}{Problem}[section]

%===============================================================================
% Graphics and Tables
%===============================================================================
\usepackage{graphicx}
\usepackage{svg}
\usepackage{subfig}
\usepackage{array}
\usepackage{longtable}
\usepackage{booktabs}
\usepackage{tabularx}
\usepackage{tikz}
\usepackage{forest}  % For tree diagrams
\usepackage{dirtree} % Alternative for directory trees

%===============================================================================
% Colors and Styling
%===============================================================================
\usepackage{xcolor}
\usepackage{color}
\usepackage{tcolorbox}

% Color definitions
\definecolor{blue}{RGB}{51,131,255}
\definecolor{applegreen}{RGB}{141,182,0}
\definecolor{orange}{RGB}{252,147,3}
\definecolor{red}{RGB}{231,24,55}
\definecolor{codegreen}{rgb}{0,0.6,0}
\definecolor{codegray}{rgb}{0.5,0.5,0.5}
\definecolor{codepurple}{rgb}{0.58,0,0.82}
\definecolor{backcolour}{rgb}{0.95,0.95,0.92}
\definecolor{commentcolor}{rgb}{0.0,0.5,0.0}
\definecolor{keywordcolor}{rgb}{0.0,0.0,0.7}
\definecolor{stringcolor}{rgb}{0.8,0.0,0.0}

%===============================================================================
% Code Listings
%===============================================================================
\usepackage{listings}

% Code listing style
\lstset{
  language=Java,
  basicstyle=\ttfamily\small,
  keywordstyle=\color{keywordcolor}\bfseries,
  commentstyle=\color{commentcolor}\itshape,
  stringstyle=\color{stringcolor},
  showspaces=false,
  showstringspaces=false,
  breaklines=true
}

%===============================================================================
% Hyperlinks and URLs
%===============================================================================
\usepackage[hyphens]{url}
\usepackage{hyperref}

% Hyperlink settings
\hypersetup{
    colorlinks=true,
    linkcolor=olive,
    filecolor=magenta,      
    urlcolor=cyan,
    pdftitle={CS1540 Algorithmic Design},
    pdfpagemode=FullScreen,
    breaklinks=true
}

%===============================================================================
% Document Structure and Formatting
%===============================================================================
\usepackage{fourier-orns}
\usepackage{fancyhdr}
\usepackage{enumitem}
\usepackage{titlesec}

% Optional: Create a matching footer rule
\renewcommand{\footrule}{%
  \hrulefill\raisebox{1.2pt}{\quad\decofourleft\decotwo\decofourright\quad}\hrulefill
  \vspace{3pt}}
\renewcommand{\footrulewidth}{0.4pt}

% Header and Footer

\pagestyle{fancy}
\lhead{BCS2120 - Intro to AI}
\rhead{06 October 2025}
\lfoot{Grégoire Mariette}
\cfoot{Page \thepage}
\rfoot{I6385935}

% Personal commands
\newcommand\y{\par \vspace{3mm}}

% Create subsubsubsection
\titleclass{\subsubsubsection}{straight}[\subsection]
\newcounter{subsubsubsection}[subsubsection]
\renewcommand\thesubsubsubsection{\thesubsubsection.\arabic{subsubsubsection}}
\renewcommand\theparagraph{\thesubsubsubsection.\arabic{paragraph}}
\titleformat{\subsubsubsection}
  {\normalfont\normalsize\bfseries}{\thesubsubsubsection}{1em}{}
\titlespacing*{\subsubsubsection}
{0pt}{3.25ex plus 1ex minus .2ex}{1.5ex plus .2ex}

\makeatletter
\renewcommand\paragraph{\@startsection{paragraph}{5}{\z@}%
  {3.25ex \@plus1ex \@minus.2ex}%
  {-1em}%
  {\normalfont\normalsize\bfseries}}
\renewcommand\subparagraph{\@startsection{subparagraph}{6}{\parindent}%
  {3.25ex \@plus1ex \@minus .2ex}%
  {-1em}%
  {\normalfont\normalsize\bfseries}}
\def\toclevel@subsubsubsection{4}
\def\toclevel@paragraph{5}
\def\toclevel@paragraph{6}
\def\l@subsubsubsection{\@dottedtocline{4}{7em}{4em}}
\def\l@paragraph{\@dottedtocline{5}{10em}{5em}}
\def\l@subparagraph{\@dottedtocline{6}{14em}{6em}}
\makeatother

\setcounter{secnumdepth}{4}
\setcounter{tocdepth}{4}

% Unicode character declarations
\DeclareUnicodeCharacter{251C}{|--}
\DeclareUnicodeCharacter{2500}{--}
\DeclareUnicodeCharacter{2502}{|--}
\DeclareUnicodeCharacter{2514}{|--}
\DeclareUnicodeCharacter{2248}{\approx}